\documentclass[11pt]{article}

% We use PSTricks to draw resolution trees -- pretty cumbersome.
% It also does not like pdflatex.
% Use: latex file.tex; dvips file.dvi; ps2pdf file.ps

\usepackage{pstricks,pst-node,pst-tree}
\newpsobject{showgrid}{psgrid}{subgriddiv=1,griddots=10,gridlabels=6pt}
\usepackage{amssymb}

\pagestyle{empty}
\textwidth	166mm
\textheight	256mm
\topmargin	-12mm
\oddsidemargin	-2mm
\evensidemargin	-2mm

\newcommand{\impl}{\mathbin{\Rightarrow}}
\newcommand{\biim}{\mathbin{\Leftrightarrow}}

\begin{document}
\begin{enumerate}
\setcounter{enumi}{43}
\item
Work through the following more substantial resolution
example in your own time
and make sure that you understand each step.
You only need to discuss this in a tutorial or the LMS Discussion 
Forum if there are steps you don't understand.

Dowsing, Rayward-Smith and Walter (see {\bf Readings Online})
give the following example of a non-trivial proof using resolution.
It is concerned with group theory.
A \emph{group} is a set endowed with a binary operation $\circ$.
If we use $P(x,y,z)$ to mean $x \circ y = z$ then we can
write the so-called \emph{group axioms} as follows:
\[
\begin{array}{lr}
   \forall x \forall y \exists z (P(x,y,z)) & \mbox{(closure)}
\\ \forall x, y, z, u, v, w
        \left([P(x,y,u) \land P(y,z,v)] \impl [P(x,v,w) \biim P(u,z,w)]\right)
   & \mbox{(associativity)}
\\ \exists x \forall y (P(x,y,y) \land \exists z (P(z,y,x)))
   & \mbox{(left identity}
\\ & \mbox{and left inverse)}
\end{array}
\]
Notice that the associativity axiom says that if $x \circ y = u$ and
$y \circ z = v$ then $x \circ v = u \circ z$.
In other words, $x \circ (y \circ z) = (x \circ y) \circ z$.
The last axiom says that there is some special element $x$ in
the set, with the property that $x \circ y = y$ for all $y$
(that is, this element is a \emph{neutral element} for $\circ$).
Moreover, for each $y$ there is a $z$ such that $z \circ y$
yields that special element
(in other words, each element has a \emph{left inverse}).
For an example of a group, consider the set $\mathbb{Z}$
of integers, endowed with the operation $+$.

We can translate the group axioms to clausal form.
The first axiom (closure) becomes
\[
  \{P(x,y,f(x,y))\}
\]
The second axiom (associativity) produces two clauses:
\[
\begin{array}{l}
   \{\neg P(x,y,u), \neg P(y,z,v), \neg P(x,v,w), P(u,z,w)\}
\\ \{\neg P(x,y,u), \neg P(y,z,v), \neg P(u,z,w), P(x,v,w)\}
\end{array}
\]

The last axiom (left identity and left inverse) also produces two clauses:
\[
\begin{array}{l}
   \{P(a,y,y)\}
\\ \{P(g(y),y,a)\}
\end{array}
\]
Suppose we want to prove that every element of a group also
has a right inverse.
That is, we want to prove
\[
  \exists x \forall y \exists z (P(y,z,x))
\]
from the axioms.
To do this we first negate our formula, obtaining:
\[
  \forall x \exists y \forall z (\neg P(y,z,x))
\]
In clausal form this becomes $\{\neg P(h(x),z,x)\}$.
On the next page is a proof by resolution.
It is a mechanical proof of a non-trivial theorem.
When there is ambiguity, I have used underlining to show which
atom takes part in the resolution step.

Make sure you understand each resolution step.
Did the refutation make use of all the axioms?
If you try to do the proof on your own without looking at the proof
above, you will find that there are many blind alleys
(most of them will end in failure due to the occur check).
So you will most likely take a long time, and do lots of back-tracking.
With a computer of course we find the refutation in a flash.

Notice how clauses have had their variables renamed to avoid name clashes.
Try to track how the variables $x'$ and $z'$ from the original
query get bound during this proof.

\pagebreak
\begin{center}
\begin{pspicture}(0,-1)(12,18) % \showgrid
\psset{nodesepA=2pt,nodesepB=2pt,linecolor=black,arrows=-,yunit=1.25cm}

\rput[c](3,14){\rnode[c]{N1}{$\neg P(x,y,u), \neg P(y,z,v), \neg P(x,v,w), P(u,z,w)$}}
\rput[c](10,14){\rnode[c]{N2}{$\neg P(h(x'),z',x')$}}
\rput[c](5,12){\rnode[c]{N3}{$\neg P(x,y,h(x')), \neg P(y,z',v), \underline{\neg P(x,v,x')}$}}
\rput[c](11,12){\rnode[c]{N4}{$P(g(u),u,a)$}}
\rput[c](7,10){\rnode[c]{N5}{$\neg P(g(u),y,h(a)), \neg P(y,z',u)$}}
\rput[c](12.3,10){\rnode[c]{N6}{$P(a,u',u')$}}
\rput[c](2.2,8){\rnode[c]{N7}{$\neg P(x_2,y_2,u_2), \neg P(y_2,z_2,v_2), \neg P(u_2,z_2,w_2), P(x_2,v_2,w_2)$}}
\rput[c](9,8){\rnode[c]{N8}{$~~\neg P(g(u'),a,h(a)$}}
\rput[c](3.9,6){\rnode[c]{N9}{$\neg P(g(u'),y_2,u_2), \neg P(y_2,z_2,a), \underline{\neg P(u_2,z_2,h(a))}$}}
\rput[c](11,6){\rnode[c]{N10}{$P(a,v',v')$}}
\rput[c](5.6,4){\rnode[c]{N11}{$\underline{\neg P(g(u'),y_2,a)}, \neg P(y_2,h(a),a)$}}
\rput[c](11,4){\rnode[c]{N12}{$P(g(u_3),u_3,a)$}}
\rput[c](7.3,2){\rnode[c]{N13}{$\neg P(u_3,h(a),a)$}}
\rput[c](12,2){\rnode[c]{N14}{$P(g(u_4),u_4,a)$}}
\rput[c](9,0){\rnode[c]{N15}{$\bot$}}

\ncline{N1}{N3}
\ncline{N2}{N3}
\ncline{N3}{N5}
\ncline{N4}{N5}
\ncline{N5}{N8}
\ncline{N6}{N8}
\ncline{N7}{N9}
\ncline{N8}{N9}
\ncline{N9}{N11}
\ncline{N10}{N11}
\ncline{N11}{N13}
\ncline{N12}{N13}
\ncline{N13}{N15}
\ncline{N14}{N15}

\rput[c](5.5,13){$\begin{array}{l}u=h(x')\\w=x'\\z=z'\enda{array}$}
\rput[c](7.5,11){$\begin{array}{l}x=g(u)\\v=u\\x'=a\end{array}$}
\rput[c](9.1,9){$\begin{array}{l}y=a\\z'=u'\\u=u'\end{array}$}
\rput[c](4.4,7){$\begin{array}{l}x_2=g(u')\\v_2=a\\w_2=h(a)\end{array}$}
\rput[c](6.2,5){$\begin{array}{l}u_2=a\\v'=h(a)\\z_2=h(a)\end{array}$}
\rput[c](7.7,3){$\begin{array}{l}u'=u_3\\y_2=u_3\end{array}$}
\rput[c](9.4,1.2){$\begin{array}{l}u_3=g(h(a))\\u_4=h(a)\end{array}$}

\end{pspicture}
\end{center}

\end{enumerate}

\end{document}
