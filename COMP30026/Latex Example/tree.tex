\documentclass[11pt]{article}
\usepackage[top=16mm, bottom=12mm, left=20mm, right=20mm]{geometry}
\pagestyle{empty}


\usepackage{tikz}
\usetikzlibrary{calc}


\title{Tikz resolution tree example}
\author{Matthew Farrugia-Roberts}
\date{August 21, 2017}


\begin{document}

% title
\begin{center}
    {\LARGE Tikz resolution tree example}\\[2ex]
    {\large Matthew Farrugia-Roberts}\\[1ex]
    {\large August 21, 2017}
\end{center}



% resolution tree
\begin{center}
\begin{tikzpicture}[scale=3, every node/.style={align=center}]
    %               '--.--'            |
    % this spaces clauses nicely.      '-> this lets us have line breaks within
    % change number to resize the tree        nodes (useful for big clauses and 
    %                                                                 unifiers)

    % clauses and layout
    % ------------------
    % state your clauses here in the format:
    % 
    %                \node (name) at (x, y) {$clause$}; <--.
    %         (x is how far left,  <--'--'               (don't forget the ';')
    %          y is how far upwards)
    \node (a) at (1, 7) {$\neg P(x, y, u), \neg P(y, z, v), \neg P(x, v, w),$
                         $P(u, z, w) $};
    \node (b) at (3, 7) {$\neg P(h(x'), z', x')$};
    \node (c) at (2, 6) {$\neg P(x,y,h(x')), \neg P(y,z',v), \neg P(x,v,x')$};
    \node (d) at (4, 6) {$P(g(u),u,a)$};
    \node (e) at (3, 5) {$\neg P(g(u),y,h(a)), \neg P(y,z',u)$};
    \node (f) at (5, 5) {$P(a,u',u')$};
    \node (g) at (4, 4) {$\neg P(g(u'),a,h(a))$};
                        % tip: use two $$ AND \\ to break a long clause in pdf:
    \node (h) at (2, 4) {$\neg P(x_2,y_2,u_2), \neg P(y_2,z_2,v_2),$\\
                         $\neg P(u_2,z_2,w_2), P(x_2,v_2,w_2)$};
                        % tip: use two $$ to break a long clause in source only:
    \node (i) at (3, 3) {$\neg P(g(u'),y_2,u_2), \neg P(y_2,z_2,a),$
                         $\neg P(u_2,z_2,h(a))$};
    \node (j) at (1, 3) {$P(a,v',v')$};
    \node (k) at (2, 2) {$\neg P(g(u'),y_2,a), \neg P(y_2,h(a),a)$};
    \node (l) at (0, 2) {$P(g(u_3),u_3,a)$};
    \node (m) at (1, 1) {$\neg P(u_3,h(a),a)$};
    \node (n) at (3, 1) {$P(g(u_4),u_4,a)$};
    \node (o) at (2, 0) {$\bot$};
    
    % edges and unifiers
    % ------------------
    % give a list of pairs of resolving clauses, their resolvent, and their 
    % unifier, in the following format:
    %                                          a     b
    %                      a/b/c/u,    --->     \ u /
    %                                            \ /
    %                                             c
    %
    % I recommend putting the unifiers in the format:
    %         $x \mapsto a$ \\ $y \mapsto b$ \\ $z \mapsto c$ \\ ...
    %
    \foreach \parentA/\parentB/\resolvent/\unifier in {
        a/b/c/$u \mapsto h(x')$\\$w \mapsto x'$\\$z \mapsto z'$,
        c/d/e/$x \mapsto g(u)$\\$v \mapsto u$\\$x' \mapsto a$,
        e/f/g/$y \mapsto a$\\$z' \mapsto u'$\\$u \mapsto u'$,
        g/h/i/$x_2 \mapsto g(u')$\\$v_2 \mapsto a$\\$w_2 \mapsto h(a)$,
        i/j/k/$u_2 \mapsto a$\\$v' \mapsto h(a)$\\$z_2 \mapsto h(a)$,
        k/l/m/$u' \mapsto u_3$\\$y_2 \mapsto u_3$,
        m/n/o/$u_3 \mapsto g(h(a))$\\$u_4 \mapsto h(a)$
    } {
        \path (\parentA) edge (\resolvent);
        \path (\parentB) edge (\resolvent);
        \node (u) at
            ($ (\parentA)!0.5!(\parentB)!0.43!(\resolvent) $)
            {\unifier};
    }
    % if you want to manually position unifiers, you should create a separate
    %         \node (u) at (x, y) {unifier};
    % outside of the above loop (and you can pass in a blank unifier to the
    % loop, so that it still gives you the edges automatically)

\end{tikzpicture}
\end{center}

\end{document}
