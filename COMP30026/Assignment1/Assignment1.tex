\documentclass[11pt]{article}
\usepackage[top = 16mm, bottom = 12mm, left=20mm, right=20mm]{geometry}
\usepackage{tikz}
\usetikzlibrary{calc}

%page setup
\pagestyle{empty}
\textwidth 166mm
\textheight 256mm
\topmargin -12mm
\oddsidemargin -2mm
\evensidemargin -2mm

%logic command setup
\newcommand{\impl}{\mathbin{\Rightarrow}}

\begin{document}
\subsection*{Challenge1}
Defining the following two proposition.
\[
\begin{array}{lr}
    a:\ A\ is\ knight
\\  b:\ B,\ C\ are\ knaves
\\  a \impl b:\ if\ A\ is\ a\ knight\ then\ his\ two\ friends\ B\ and\ C\ here\ are\ knaves
\end{array}
\]
We then generating the truth table as follows:
\[
\begin{array}{lr}
\\  a\ \ \ \ \ \ \ \ \ \ \ \ a \impl b\ \ \ \ \ \ \ \ \ \ \ \ b\ \ \ \ \ \ Follow\ logic\ rule?
\\  0\ \ \ \ \ \ \ \ \ \ \ \ \ \ \ 0 \ \ \ \ \ \ \ \ \ \ \ \ \ \ \ 0\ \ \ \ \ \ \ \ No
\\  0\ \ \ \ \ \ \ \ \ \ \ \ \ \ \ 0 \ \ \ \ \ \ \ \ \ \ \ \ \ \ \ 1\ \ \ \ \ \ \ \ No
\\  1\ \ \ \ \ \ \ \ \ \ \ \ \ \ \ 1 \ \ \ \ \ \ \ \ \ \ \ \ \ \ \ 0\ \ \ \ \ \ \ \ No
\\  1\ \ \ \ \ \ \ \ \ \ \ \ \ \ \ 1 \ \ \ \ \ \ \ \ \ \ \ \ \ \ \ 1\ \ \ \ \ \ \ \ Yes
\end{array}
\]
From the table we have the following situation:

if $a$ is false, that means A is not a knight. So he has to tells lies. This means the statement $a \impl b$ has to be false. However if $a$ is false then $a \impl b$ cannot be false in any case. Therefore, $a$ cannot be false, so A has to be a knight.

Now Since A is a knight, $a \impl b$ has to be true, because knight always tells the truth. Since $a$ and $a \impl b$ are both true, statement $b$ has to be true.\\
Therefore, B and C are Knaves. A is a knight.


\subsection*{Challenge 2}
\begin{enumerate}
\item
The CNF of $\varphi$ is:
\[
\begin{array}{lr}
    \neg(((P \impl S) \land (Q \impl R) \land (R \impl P) )\impl S))
\\  \neg(\neg((P \impl S) \land (Q \impl R) \land (R \impl P) ) \lor S)
\\  ((\neg P \lor S) \land (\neg Q \lor R) \land (\neg R \lor P)) \land \neg S)
\\  \neg P \land \neg S \land (\neg Q \lor R) \land (\neg R \lor P)
\\  \neg P \land \neg S \land (\neg Q \lor R) \land \neg R
\\  \neg P \land \neg S \land \neg Q \land \neg R
\end{array}
\]
\item
$\varphi$ is non-valid by ssigning $P, S, R, Q$ to False.

So the formula would become:
\[
\begin{array}{lr}
    ((F \impl F) \land (F \impl F) \land (F \impl F)) \impl F
\\  (T \land T \land T \impl F)
\\  False
\end{array}
\]
Therefore, $\varphi$ is non-valid as it can be made to False.
\item
$\neg \psi$ in CNF form is:
\[
\begin{array}{lr}
    \neg((((P \lor Q) \impl S) \land (\neg P \impl (R \impl Q)) \land (R \lor S)) \impl S)
\\  \neg(\neg(((P \lor Q) \impl S) \land (\neg P \impl (R \impl Q)) \land (R \lor S)) \lor S)
\\  ((P \lor Q) \impl S) \land (\neg P \impl (R \impl Q)) \land (R \lor S)) \land \neg S
\\  (\neg(P \lor Q) \lor S) \land
    (P \lor (\neg R \lor Q)) \land
    (R \lor S) \land \neg S
\\  (\neg P \land \neg Q \land \neg S) \land
    (P \lor (\neg R \lor Q)) \land
    (R \lor S)
\\  \neg P \land \neg Q \land \neg S \land
    (P \lor \neg R \lor Q) \land R
\end{array}
\]
\newpage
\item
We prove $\psi$ is valid by deriving $\bot$ using refutation on $\neg \psi$:

The refutation tree is as followed:
\begin{center}
\begin{tikzpicture}[scale=3, every node/.style={align=center}]
	%put up the tree node element
	\node (a) at (0,3) {$P \lor Q \lor \neg R$};
	\node (b) at (2,3) {$\neg P$};
	\node (c) at (1,2) {$\neg R \lor Q$};
	\node (d) at (3,2) {$\neg R$};
	\node (e) at (2,1) {$Q$};
	\node (f) at (4,1) {$\neg Q$};
	\node (g) at (3,0) {$\bot$};
	%set up the edge here
	\foreach \parentA/\parentB/\resolvent/\unifier in {
        a/b/c/,
        c/d/e/,
        e/f/g/
    } {
        \path (\parentA) edge (\resolvent);
        \path (\parentB) edge (\resolvent);
        \node (u) at
            ($ (\parentA)!0.5!(\parentB)!0.43!(\resolvent) $)
            {\unifier};
    }
\end{tikzpicture}
\end{center}
Therefore, from refutation tree, we know $\psi$ is valid.
\end{enumerate}

\vskip 1cm
\subsection*{Challenge3}
The non-valid case is when we set following interpretation to the formula:
\[
\begin{array}{lr}
    P(x,y):\ x\ does\ not\ equal\ to\ y
\\  h(x):\ return\ x\ itself
\end{array}
\]
with the domain of x and y to be:
\[
\begin{array}{lr}
    x's\ domain\ is\ all\ even\ number\ in\ R
\\  y's\ domain\ is\ all\ odd\ number\ in\ R
\end{array}
\]
So the equation would become:
\[
\begin{array}{lr}
    [(False \impl False) \impl (False \land False)]
\\  True \impl Flase \equiv False
\end{array}
\]
Now the Satisfiable case is we set $P(x,y) = True$, whatever $x$ and $y$ is.So the formula would become
\[(True \impl True) \impl (True \land True)\]
 which is True. This means the formula is satisfiable.\\
Therefore the formula is non-valid but satisfiable.
\newpage
\subsection*{Challenge4}
\begin{enumerate}
\item
$S_1:\ \forall x\ (S(x)\land \forall y (\neg P(y,x))) \impl H(x)$
\item
$S_2:\ \forall x\ (S(x) \land (\forall y (P(y,x) \impl R(y)))) \impl H(x)$
\item
$S_2$ in CNF form is:
\[
\begin{array}{lr}
    \forall x\ \neg(S(x) \land
    (\forall y (P(y,x) \impl R(y)))) \lor H(x)
\\  \forall x\ \neg S(x) \lor
    \neg(\forall y (\neg P(y,x) \lor R(y)))) \lor H(x)
\\  \forall x\ \neg S(x) \lor
    \exists y (P(y,x) \land \neg R(y))
    \lor H(x)
\\  \forall x\ \neg S(x) \lor
    (P(f(x),x) \land \neg R(f(x))) \lor
    H(x)
\\  \forall x\ 
    (\neg S(x) \lor H(x) \lor P(f(x),x)) \land
    (\neg S(x) \lor H(x) \lor \neg R(x))
\end{array}
\]
Therefore, the clausal form is:
\[
    \{\{\neg S(x) \lor H(x) \lor P(f(x),x\},
    \{\neg S(x) \lor H(x) \lor \neg R(x)\}\}
\]
\item
$\neg S_1$ in CNF form is:
\[
\begin{array}{lr}
    \neg (\forall x\ (S(x)\land \forall y (\neg P(y,x))) \impl H(x))
\\  \neg (\forall x\ \neg(S(x)\land \forall y (\neg P(y,x))) \lor H(x))
\\  \exists x\ S(x) \land \forall y \neg P(y,x) \land \neg H(x)
\\  S(a) \land \neg P(y,a) \land \neg H(a)
\end{array}
\]
Therefore, the clausal form is :
\[
    \{\{S(a)\},\{\neg P(y,a)\},\{\neg H(a)\}\}
\]
\item
We now refutate $S_2 \land \neg S_1$ to get $\bot$, thus prove $S_1$ is a logical Consequence of  $S_2$:
`
The refutation tree is as follows:
%resolution tree
\begin{center}
\begin{tikzpicture}[scale=3, every node/.style={align=center}]
	%put in tree here
	\node (a) at (0,4) {$\neg S(x) \lor H(x) \lor P(f(x),x)$};
	\node (b) at (2,4) {$S(a)$};
	\node (c) at (1,3) {$H(a) \lor P(f(a),a)$};
	\node (d) at (3,3) {$\neg H(a)$};
	\node (e) at (2,2) {$P(f(a),a)$};
	\node (f) at (4,2) {$\neg P(y,a)$};
	\node (g) at (3,1) {$\bot$};
	\node (h) at (1,1) {$\neg S(X) \lor H(x) \lor \neg R(x)$};
	\node (i) at (2,0) {$\bot$};
	\foreach \parentA/\parentB/\resolvent/\unifier in {
        a/b/c/$x \mapsto a$,
        c/d/e/,
        e/f/g/$y \mapsto f(a)$,
        g/h/i/
    } {
        \path (\parentA) edge (\resolvent);
        \path (\parentB) edge (\resolvent);
        \node (u) at
            ($ (\parentA)!0.5!(\parentB)!0.43!(\resolvent) $)
            {\unifier};
    }

\end{tikzpicture}
\end{center}
\end{enumerate}

\end{document}
